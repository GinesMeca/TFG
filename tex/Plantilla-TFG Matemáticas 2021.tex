\documentclass[12pt,twoside]{article}

% La extensión total de la memoria deberá ser de un máximo de 50 páginas (excluidos resumen, índice y posibles anexos).

% Según las recomendaciones de estilo, el formato de la memoria se ajustará a lo siguiente:
% - Formato del papel: DIN A4.
% - Impresión a dos caras.
% - Márgenes: superior e inferior, 2.5 cm. Márgenes laterales: páginas impares, izquierdo 4 cm y derecho 2 cm; páginas pares, izquierdo 2 cm y derecho 4 cm.
% - Tipo de letra: Times New Roman de 12 puntos.
% - Interlineado: 1.5 líneas.
% - Alineación: justificación completa.
% - Sangrado de párrafo: 0.5 cm la primera línea de cada párrafo. No se pondrá espacio entre párrafos.
% - Las páginas deberán ir numeradas en números arábigos.

% Teniendo en cuenta las indicaciones previas, definimos el estilo en LaTeX:

% Indicaciones para el idioma:
\usepackage[T1]{fontenc}
\usepackage[utf8]{inputenc}
\usepackage[spanish]{babel}

% Adaptación de itemize y enumerate a los usos tipograficos españoles:
\let\layoutspanish\relax
\addto\captionsspanish{\def\tablename{Tabla}} % para que escriba "Tabla" en lugar de "Cuadro"
\unaccentedoperators  % para que no acentúe los operadores

% Área de impresión de una página:
\usepackage[a4paper]{geometry}
  \geometry{hmargin={2.5cm,2.5cm},height=22cm}

% Formato de algunas distancias:
\renewcommand{\baselinestretch}{1.2}    % separación entre líneas de un mismo párrafo
\setlength{\partopsep}{0pt}
\setlength{\itemsep}{0pt}
\setlength{\topsep}{0pt}
\setlength{\parsep}{0pt}
\setlength{\parskip}{0.25\baselineskip}   % separación entre párrafos

\renewcommand{\textfraction}{0.1}   % mínima fracción de la página para el texto
\renewcommand{\topfraction}{1}      % máxima fracción de la página para objetos flotantes en la parte superior
\renewcommand{\bottomfraction}{1}
\renewcommand{\floatpagefraction}{1}

\setcounter{totalnumber}{5}
\setcounter{topnumber}{3}
\setcounter{bottomnumber}{2}

% Adaptación de las "caption" de los entorns "figure" y "table":
\usepackage{caption}

% Indentación del primer párrafo de una sección:
\usepackage{indentfirst}

% Definición del color grisclaro en la salida PDF:
\usepackage[pdftex]{color}

% Gráficos:
\usepackage[pdftex]{graphicx}

% Paquetes recomendados para la inclusión de fórmulas matemáticas:
\usepackage{amsmath}
\allowdisplaybreaks  % para que pueda partir fórmulas que ocupan más de una línea, necesita el paquete anterior
\usepackage{amssymb} % para cargar algunos símbolos como \blacksquare y \square
\usepackage{amsfonts} % para cargar algunas fuentes en estilo matemático
\usepackage{enumerate}
% Teoremas (se pueden definir todos los que se necesiten):

\newtheorem{theorem}{Teorema}[section]
\newtheorem{proposition}[theorem]{Proposición}
\newtheorem{definition}[theorem]{Definición}
\newtheorem{lemma}[theorem]{Lema}
\newtheorem{corollary}[theorem]{Corolario}
\newtheorem{example}[theorem]{Ejemplo}
\newtheorem{app}[theorem]{Aplicación}
\newtheorem{remark}[theorem]{Observación}
\newtheorem{agrad}[theorem]{Agradecimiento}
\newtheorem{algo}[theorem]{Algoritmo}
\newtheorem{axiom}[theorem]{Axioma}
\newtheorem{case}[theorem]{Caso}
\newtheorem{conclu}[theorem]{Conclusión}
\newtheorem{conjectura}[theorem]{Conjetura}
\newtheorem{notac}[theorem]{Notación}
\newtheorem{soluc}[theorem]{Solución}
\newtheorem{summary}[theorem]{Sumario}


\newtheorem{proof}[theorem]{Demostración.}
\renewenvironment{proof}{\emph{Demostración.}} {\quad \hfill $\blacksquare$ \newline} % para que aparezca un cuadrado negro al acabar la demostración


% Definición de cabeceras y pies de página:

\usepackage{fancyhdr}                     % para definir distintos tipos de cabeceras y pies de página

\newcommand{\RunningAuthor}{Nombre Apellido1 Apellido2}
\newcommand{\Author}[1]{\renewcommand{\RunningAuthor}{#1}}
\renewcommand{\leftmark}{\RunningAuthor}

\newcommand{\RunningTitle}{Trabajo de fin de grado}
\newcommand{\Title}[1]{\renewcommand{\RunningTitle}{#1}}
\renewcommand{\rightmark}{\RunningTitle}

\pagestyle{fancy}
\fancyhf{}
\fancyhead[LO]{\small \slshape \leftmark}    % lo que aparece en la parte izquierda de la páginas impares
\fancyhead[RE]{\small \slshape \rightmark}   % lo que aparece en la parte derecha de las páginas pares
\fancyhead[RO,LE]{\small \slshape \thepage}  % el número de página aparece en la parte exterior de la cabecera

\renewcommand{\headrulewidth}{0.6pt}         % grueso de la línea horizontal por debajo de la cabecera de la página
\renewcommand{\footrulewidth}{0pt}           % grueso de la línea horizontal por encima del pie de página
                                             % en este caso está vacío
\setlength{\headheight}{1.5\headheight}      % aumenta la altura de la cabecera en una parte y media

\fancypagestyle{plain}{%                     % redefinición del estilo de página 'plain'
  \fancyhf{}                                 % limpia todas las cabeceras y pies de página
  \setlength{\headwidth}{\textwidth}
  \fancyfoot[C]{\small \slshape \thepage}    % excepto el centro del pie de página
  \renewcommand{\headrulewidth}{0pt}
  \renewcommand{\footrulewidth}{0pt}
  }

% Instrucciones que se usan frecuentemente
\newcommand{\abs}[1]{\ensuremath{|#1|}}

% Datos del trabajo y autor:
\title{Título}
\author{Nombre Apellido1 Apellido2\\*[1em]
\begin{minipage}{0.75\textwidth}
\footnotesize \itshape
\begin{center}
Universidad de Alicante \\
4º de Grado en Matemáticas
\end{center}
\end{minipage}
}
\date{Julio 2021}

% Para incluir paginas de otro pdf (por ejemplo, la de la portada):
\usepackage{pdfpages}


\begin{document}

% Para introducir la portada en castellano, se guardar anexo-1-portada-memoria-tfg-matematicas.pdf en el mismo directorio:
\includepdf[pages=1]{anexo-1-portada-memoria-tfg-matematicas.pdf}


% Después de la portada, se introducirá un resumen del Trabajo Fin de Grado (máximo 500 palabras) en una de las lenguas oficiales y en inglés, junto con las palabras clave (de 3 a 5).

\section*{Resumen}


\emph{Resumen del Trabajo Fin de Grado (máximo 500 palabras) en una de las lenguas oficiales}



[...]

\newpage

\section*{Abstract}

\emph{Resumen del Trabajo Fin de Grado (máximo 500 palabras) en inglés}

[...]

% A continuación, se incluirá el índice del trabajo y, seguidamente, se desarrollará la memoria.
\newpage

\tableofcontents

\newpage

\section{Introducción}\label{sec:1}

{\color{red}
La introducción describe el tema de interés desde un punto de vista global explicitando el objetivo del trabajo, las definiciones necesarias y haciendo referencias a resultados previos si fuera necesario. También se puede explicar cómo se ha estructurado el trabajo.}

\section{Métodos}

{\color{red}
En este apartado/s se desarrollan los métodos que se van a utilizar posteriormente o no (en el caso de que el trabajo fuese teórico o de revisión bibliográfica). Podrá tener una o varias secciones y el título no tiene porqué ser 'Métodos'.}

Aprovechamos este espacio para hablar de referencias.

\textbf{1. Referencias a gráficas y tablas.}

Todas las tablas y figuras han de estar enumeradas. Para ello hay que poner dentro del entorno figure o table \textbackslash label y usar \textbackslash ref en el texto para explicarlo. Ejemplos:

\begin{figure}[h]
\centerline{\includegraphics[scale=0.5]{UA.jpg}}
\caption{Logo de la UA}\label{fig:01}
\end{figure}

\begin{figure}[h]
\centerline{\includegraphics[scale=0.5]{ciencias.jpg}}
\caption{Logo de la UA}\label{fig:02}
\end{figure}

En la Figura \ref{fig:01} se puede ver el logo de la UA y en la Figura \ref{fig:02} el de la Facultad de Ciencias.


\begin{table}[ht] 
\centering
\begin{tabular}{rrrr} 
  \hline
 & setosa & versicolor & virginica \\ 
  \hline
setosa &  50 &   0 &   0 \\ 
  versicolor &   0 &  48 &   2 \\ 
  virginica &   0 &   1 &  49 \\ 
   \hline
\end{tabular}
\caption{Resultados de LDA} \label{tab:01}
\end{table}

La Tabla \ref{tab:01} muestra los resultados de clasificación con LDA frente a valores reales.

\textbf{2. Referencias a fórmulas o resultados previos}

Se pondrá label dentro del entorno de la ecuación.

\begin{equation}
\begin{array}{rcl}
a^tx_i &<& c, \hspace{0.2cm} \tiny{i=1,...,n_1} \\
b^ty_j &<& d, \hspace{0.2cm} \tiny{j=1,...,n_2}
\end{array} \label{eq:1}
\end{equation}

Sean $\varepsilon_i$ y $\delta_j$, las holguras de (\ref{eq:1}).

\textbf{3. Referencias a secciones, capítulos...}

Se pondrá label al lado del título.

Como ya se dijo en la sección \ref{sec:1} en este apartado... En la sección \ref{sec:2} veremos ... 


\textbf{4. Referencias bibliográficas.}

Todas las referencias bibliográficas que aparecen al final han de estar mencionadas en el texto. Para ello, cada \textbackslash bibitem tendrá un CÓDIGO entre llaves. Hay que usar \textbackslash cite \{CÓDIGO\} en el momento que lo citéis.

En \cite{FA03} podemos ver que ... 
Podemos encontrar otros resultados similares en \cite{AR01} y \cite{CA01}.


\section{Resultados} 
{\color{red}
En el caso de trabajos que tengan una parte práctica, en este apartado/s se exponen los resultados de la aplicación de los métodos a casos concretos. Podrá tener una o varias secciones y el título no tiene porqué ser 'Resultados'.}
\subsection{Estudio 1} \label{sec:2}

\section{Conclusiones}
{\color{red}
Las conclusión/discusión final ha de responder de forma clara a los objetivos del trabajo a partir de los resultados obtenidos y dejar constancia del potencial o limitaciones del trabajo mostrando capacidad de espíritu crítico y autocrítico.}

%-------------------------------------------------------------------------------------------------------
\addcontentsline{toc}{section}{Referencias}
\begin{thebibliography}{100}




% Ejemplo de referencia de un libro:
\bibitem{FA01} Fajardo, M.D., Goberna, M.A., Rodríguez, M.M.L. and Vicente-Pérez, J. (2020).
\textit{Even Convexity and Optimization: Handling Strict Inequalities}. Springer.

% Ejemplo de referencia de un capítulo de un libro:
\bibitem{AR01} Aragón, F.J., Convexity in Nonlinear Optimization in Aragón, F.J., Goberna, M.A., López, M.A. and Rodríguez, M.L. (2019) pp.55-89.

% Ejemplo de referencia de artículos:
\bibitem{AL01} Alonso-González, C., Navarro-Pérez, M.A. and Soler-Escrivà, X. (2020). \textit{Flag codes from planar spreads in network coding. Finite Fields and their applications} \textbf{68}, 101745.


% Ejemplo tesis doctorales:
\bibitem{CA01} Campoy, R. (2018) Contributions to the Theory and Applications of Projection Algorithms. Tesis doctoral Universidad de Murcia, España.

% Ejemplo páginas web:
\bibitem{LE01} LeCun, Y., Cortes, C. and Burges, C.J.C., The MNIST database of handwritten digits. http://yann.lecun.com/exdb/mnist/ (Consultado el 25 de Junio de 2021).


\end{thebibliography}


%-------------------------------------------------------------------------------------------------------
\newpage
\appendix

\section{Detalles del desarrollo del trabajo}
{\color{red} En un anexo se incluirán siguientes tablas. 
\begin{enumerate}
\item  En la primera tabla se especificarán las tareas más relevantes desarrolladas y el tiempo aproximado en horas dedicado. Recordad que el TFG es una asignatura de 6 créditos, por lo que el tiempo de dedicación esperado debería ser de 150 horas. La tabla \ref{tab{02}} muestra a modo de ejemplo algunas tareas.
\item En la segunda tabla se especificarán las asignaturas con las que el trabajo está relacionado, detallando dicha relación. La tabla \ref{tab{03}} muestra algunos ejemplos.
\end{enumerate}

\begin{table}[ht] 
\centering
\begin{tabular}{lc} 
  \hline
 Tarea & Tiempo (horas) \\ 
  \hline
Recopilación de materiales &   ... \\ 
Estudio de bibliografía &   ... \\ 
Elaboración de resultados gráficos/numéricos &  ... \\ 
Redacción de la memoria &  ... \\
 ... &  ... \\
 ... &  ... \\
 \hline
Total & 150\\
\hline
\end{tabular}
\caption{Tiempo aproximado de dedicación al trabajo} \label{tab{02}}
\end{table}

\begin{table}[ht] 
\centering
\begin{tabular}{llll} 
  \hline
 Asignatura & Páginas & Descripción  \\ 
  \hline
Asignatura1 & 15-17 & Apdo 2.1 se obtiene a partir de los apuntes \\ 
Asignatura2 & ...  & El teorema 1 y su demostración es material del profesor\\ 
Asignatura3 & 38 & Hay relación, pero no es referencia básica \\ 
... & ... & ...\\ 
\hline
\end{tabular}
\caption{Asignaturas relacionadas con el trabajo} \label{tab{03}}
\end{table}

\end{document} 